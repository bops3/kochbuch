\begin{recipe}
[ % Optionale Eingaben
    preparationtime = {\unit[30]{min}},
    bakingtime={\unit[15]{min}},
    bakingtemperature={\Fanoven\ \unit[250]{°C}},
    portion = \portion{2},
    %calory,
    source = Lynn
]
{Flammkuchen}

\introduction{
    \# Minimaler Einkaufsaufwand
}

\graph{
    small = lynn/flamm1,
    big = lynn/flamm2,
}

\ingredients
{% Zutatenliste
    \unit[250]{g} & Mehl \\
    \unit[125]{mL} & Wasser \\
    Etwas  & Salz \\
    \unit[3]{EL} & Öl \\
    \unit[125]{g} & Speck \\
    oder \\
    eine Hand voll & Champignons \\
    2 & rote Zwiebeln \\
    \unit[250]{g} & Créme Fraîche
}

\preparation
{ % Schrittweise Zubereitung
    \\
    Die Zutaten für den Teig zusammengeben und etwas kneten. In 2 Teile teilen und sehr dünn ausrollen. Auf 2 Blechen verteilen.
    
    Die Creme Fraiche auf dem Teig verteilen und mit Salz und Pfeffer würzen. Zwiebeln in Halbringe schneiden und mit einem Schluck Wasser 1 Minute bei ~500 W in Mikrowelle geben, damit die Zwiebeln im Ofen nicht so leicht anbrennen. 
    
    Zwiebeln und Speck oder Pilze (oder beides) verteilen und die Flammkuchen 15 Minuten bei 250 °C backen.
}

\hint
    {% Hinweise
    Die Erfahrung zeigt: Eine Person ist durchaus in der Lage ein Blech alleine zu essen…

    Alternative Belagmöglichkeiten sind: Tomaten, Oliven und Käse oder Zwiebeln, Feta und Honig.
    
    Auch diese Bilder sind von einer KI.
    }

\end{recipe}
