\begin{recipe}
[ % Optionale Eingaben
    preparationtime = {\unit[30]{min}},
    bakingtime={\unit[50]{min}},
    bakingtemperature={\Fanoven\ \unit[180]{°C}},
    portion = \portion{12},
    %calory,
    source = Lynn
]
{Käsekuchen}

\introduction{
    \# kein Käse enthalten
}

\graph{
    small = lynn/kuchen1,
    big = lynn/kuchen2,
}

\ingredients
{% Zutatenliste
    \unit[200]{g} & Mehl \\
    \unit[75]{g} & Zucker \\
    1 & Ei \\
    \unit[1]{TL} & Backpulver \\
    \unit[500]{g} & Quark \\
    2  & Eier \\
    \unit[140]{g} & Zucker \\
    \unit[1]{Pck.} & Vanillepudding \\
    \unit[100]{g} & Saure Sahne \\
    \unit[80]{mL} & Öl \\
    \unit[240]{mL} & Milch \\
    \unit[1]{Dose} & Mandarinen \\
    \unit[1]{Pck.} & Tortenguss 
}

\preparation
{ % Schrittweise Zubereitung
    \\
    Für den Teig Mehl, 75\,g Zucker, ein Ei und Backpulver mit dem Rührer mixen und auf dem Boden einer runden Kuchenform ausbreiten und fest drücken. 
    
   Quark, zwei Eier, 140\,g Zucker, Vanillepudding, Saure Sahne, Öl und Milch in einer Schüssel zusammenkippen und verrühren.
   
   Den Belag auf den Boden in der Form gießen. Mandarinen darauf verteilen und 50 Minuten bei 180 °C Umluft backen. Nach dem Backen den Tortenguss nach Anleitung anrühren und auf dem Kuchen verteilen.
}

\hint
    {% Hinweise
    Schmeckt besonders gut solange der Kuchen noch warm ist.
    
    Und Überraschung: auch diese Bilder sind von einer KI.
    }

\end{recipe}
