
\begin{recipe}
[ % Optionale Eingaben
    preparationtime = {\unit[15]{min}},
    bakingtime={\unit[20]{min}},
    bakingtemperature={\Fanoven\ \unit[180]{C}},
    portion = \portion{4},
    %calory,
    source = Malte
]
{Rote Bete mit Feta und Cashews}

\introduction{
    \# Interessante Kombi
}

\graph{
    small = malte/b1,
    big = malte/b2,
}

\ingredients[7]
{% Zutatenliste
    \unit[500]{g} & Rote Bete \\
    2 & Rote Zwiebeln \\
    \unit[200]{g} & Feta- oder Ziegenkäse \\
    \unit[150]{g} & Cashewkerne \\
    \unit[3]{TL} & Honig \\
    \unit[2]{EL} & Öl
}

\preparation
{ % Schrittweise Zubereitung
    \\
    Rote Bete kochen und schälen oder gleich vorbereitete Kaufen.

    Rote Bete in größere Häppchen schneiden.
    Zwiebeln in große Stücke schneiden (\nicefrac{1}{8} oder so).
    Beides gemischt in eine Auflaufform verteilen, würzen und das Öl untermischen.

    Im Ofen bei \unit[180]{C} kurz für ca. 5min backen.

    Cashews zerkleinern und darüber streuen.
    Darauf den Honig verteilen.

    Wieder in den Backofen schieben, bis der Käse und die Cashews leicht braun werden.

    Zusammen mit einer Beilage wie Baguettes oder Reis servieren.
}

\hint
    {% Hinweise
    Kann auch sehr gut kalt gegessen werden.

    Die Bilder wurde von einer glücklichen KI generiert.
}

\end{recipe}
