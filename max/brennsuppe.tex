

\begin{recipe}
	[ % Optionale Eingaben
	preparationtime = {\unit[15]{min}},
	source = {Von Max' Oma}
	]
	{Brennsuppe}
	
	\introduction{
		\# Noch eine Suppe, gab es früher als Beilage zum Abendessen.
	}
	
	\graph{
		big = max/brennsuppe1.jpeg,
		small = max/brennsuppe2.jpeg,
	}
	
	\ingredients
	{% Zutatenliste
		\unit[1]{EL} & erhitzbares Fett\\
		\unit[3]{EL} & Grieß\\
		\unit[$\approx 500$]{ml} & Wasser\\
		&Gemüsebrühe\\
		1 & Ei (oder auch zwei?)\\
	}
	
	\preparation
	{\\
		Das Fett in einem kleinen Topf flüssig werden lassen und darin bei moderater Hitze (es sollte nicht rauchen...) den Grieß unter häufigem Umrühren einige Minuten anrösten, bis er leicht braun wird. 
		
		Nach hinreichender Röstung Topf vom Herd nehmen und mit \textbf{nicht  kochendem} Wasser ablöschen. (Vorsicht, kann spritzen!) Entsprechend dosierte Gemüsebrühe hinzufügen. Dann 5--10 Minuten köcheln und den Grieß quellen lassen.
		
		Zuletzt das Ei/die Eier schaumig rühren und unter Umrühren in die gerade nicht mehr kochende Suppe gießen. Ggf. dann noch einmal kurz aufkochen.
	}
	
	\hint
	{% Hinweise
		Den Grieß mit Butter anzurösten schmeckt zwar super, aber ist vermutlich ungesund. Als Alternative schmeckt auch Ghee gut, das höher erhitzbar ist.
		
		Bei mehr Hunger gehen natürlich auch immer noch Suppennudeln, aber dann wird es möglicherweise ziemlich dickflüssig.
		
		Die Konsistenz des Eis lässt sich stark durch die Temperatur beim Einrühren variieren, viel Erfolg beim Ausprobieren! Meine Oma fand immer etwas größere "`Flocken"' ideal, für die das Wasser ganz kurz abkühlen muss.
		
		Möge die Küche nicht zum ersten Bild mutieren. Die KI hat wohl schlecht geträumt.
	}
	
\end{recipe}