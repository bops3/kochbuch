
\begin{recipe}
[ % Optionale Eingaben
    preparationtime = {\unit[15]{min}},
    portion = \portion{4},
    source = Chris' Mitbewohner David
]
{Hirsesalat}

\introduction{
    \# High Protein 
    \# Flexibel Einsetzbar
    \# Macht sich gut bei jedem Grillbuffet
}

\graph{
    big = chris/Hirse
}

\ingredients
{% Zutatenliste
    \unit[250]{g} & Hirse \\
    \unit[250]{g} & Feta \\
    4 & Tomaten \\
    1 & Gurke \\
    1 & kl. Dose Mais \\
    1 & kl. Dose rote Bohnen \\
    & etwas Balsamico \\
}

\preparation
{ % Schrittweise Zubereitung
    \\
    Hirse mit dem doppelten Volumen Wasser in einen Topf geben.
    Einmal aufkochen und anschließend abgedeckt auf sehr niedriger Stufe köcheln lassen.
    Die Hirse ist fertig sobald das Wasser vollständig aufgenommen oder verdampft wurde. 
    In der Zwischenzeit das Gemüse und den Feta würfeln und die Dosen abgießen.
    Wenn die Hirse fetig ist, alles in einer großen Schüssel vermischen und mit Salz, Pfeffer und Balsamico Essig abschmecken.}

\hint
    {% Hinweise
    Falls du den Salat für mehrere Tage gekocht hast, gib den Balsamico immer nur an das was du direkt essen möchtest, die Tomaten werden sonst schneller schlecht. 
    Wenn du das berücksichtigst, hält sich der Salat aber problemlos für 3-4 Tage im Kühlschrank.
    }
\end{recipe}
