
\begin{recipe}
[ % Optionale Eingaben
    preparationtime = {\unit[10]{min}},
    bakingtime={\unit[20]{min}},
    bakingtemperature={\Fanoven\ \unit[180]{C}},
    portion = \portion{2},
    %calory,
    source = Chris' Freund Paul
]
{Gemüse mit Gnocchi}

\introduction{
    \# Am Ende des Geldes noch viel Monat übrig aber keine Lust auf Nudeln
    \# Easy Peasy
    \# Veggi
}

\graph{
    big = chris/gnocchi,
}

\ingredients
{% Zutatenliste
    \unit[250]{g} & Feta \\
    1 & Rote Zwiebel \\
    2 & Rote Tomaten \\
    2 & Grüne Zucchini \\
    etwas & Knoblauch \\
    \unit[1]{Pck.} & Gnocchi \\
}

\preparation
{ % Schrittweise Zubereitung
    \\
    Backofen auf 180°C vorheizen.
    Zucchinis, Tomaten, Knoblauch und Zwiebeln würfel und mit den Gnocci vermischen.
    Mit Salz und Pfeffer würzen und auf einem Backblech ausbreiten. 
    Macht nix, wenn es sich etwas stapelt.
    Backblech in den Ofen legen.
    Nach etwa 10 Minuten den Feta drüber bröseln und nochmal backen bis der Feta etwas knusprig wird.
}


\end{recipe}
