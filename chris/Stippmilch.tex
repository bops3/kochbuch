
\begin{recipe}
[ % Optionale Eingaben
    preparationtime = {\unit[5]{min}},
    portion = \portion{4},
    source = Chris' Oma
]
{Stippmilch}

\introduction{
    \# Münsterländer Original
    \# Schnell gemacht
    \# Mit dem extra an Calcium
}

\graph{
    small = chris/Stippmilch1,
    big = chris/Stippmilch2,
}

\ingredients
{% Zutatenliste
    \unit[500]{g} & Magerquark \\
    etwas & Milch (Kuh oder Hafer) \\
    etwas & Zucker (1-5 Esslöffel) \\
    & Toppings nach Belieben \\
}

\preparation
{ % Schrittweise Zubereitung
    \\
    Magerquark in einer Schüssel mit der Milch verrühren, sodass eine glatte Maße entsteht. 1-5 Esslöffel Zucker hinzugeben und verrühren. 
    Die Stippmilch kann mit frischem Obst, Schokostreußeln, Kakao, Fruchtsirup etc. gegessen werden.
    Meine Lieblingsvariante ist mit frischen Erdbeeren. }
\end{recipe}
