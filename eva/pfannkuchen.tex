\begin{recipe}
[ % Optionale Eingaben
    preparationtime = {\unit[10-80]{min}, abhängig von der Ofenqualität und Menge},
    %bakingtime={\unit[6]{h}},
    %bakingtemperature={\Fanoven\ \unit[170]{C}},
    %portion = \portion{8},
    %calory,
    source = Eva
]
{Pfannkuchen}

\introduction{
    \# Das einzige Rezept, das ich kann
}

\graph{
    small = eva/pfannkuchen_durchschnittlich, %offensichtliches Bild auf dem Desktop 1,
    big = eva/pfannkuchen_fancy %offensichtliches Bild auf dem Desktop 2,
}

\ingredients
{% Zutatenliste
    \unit[\nicefrac{1}{2}]{Einheit} & Mehl, z.B. \unit[500]{g} \\
    \unit[1]{Einheit} & Milch, z.B. \unit[1]{l} \\
    \unit[1] & Ei \\
    wenig & Backpulver \\
    genug & Zimt \\
    beliebiger & Belag \\
}

\preparation
{ % Schrittweise Zubereitung
    \\
    Man nehme doppelt so viel Milch wie Mehl und ein Ei pro \unit[500]{g} Mehl. Oder einfach immer ein Ei.
    Oder so viele, wie man grade da hat und loswerden möchte. Und mische es zusammen.

    Gute Ergänzungen sind etwas Backpulver oder Zimt. Auch davon kann jeweils beliebig viel hinzugefügt werden.
    Es empfiehlt sich ungefähr eine Prise Backpulver und eine Schüsseloberfläche voll Zimt. Evtl. kann auch eine Prise
    Salz hinzugefügt werden.

    Anschließend mischt man alles gut mit einem Schneebesen oder Rührgerät durch - es kann auch erst lange 
    umgerührt, dann der Zimt hinzugefügt und dann nochmal kurz umgerührt werden - und erhitzt den Teig passend zur 
    zur Verfügung stehenden Pfanne portioniert in einer Pfanne.
}

\hint
    {% Hinweise
    Erinnerst du dich, dass wir uns irgendwann darüber unterhielten, welches Essen dasjenige sei,
    von dem man sich im Zweifelsfall am ehesten ausschließlich ernähren könnte? Ich glaube, Brot wäre gut. Aber solltest
    du dich doch mal gegen Suppe entscheiden, hast du jetzt dieses nützliche Rezept :)
    
    Der Belag ist wirklich sehr frei wählbar. Ich nehme oft Apfelmus mit mehr Zimt, oder nichts oder Marmelade. Frevlerische 
    Individuen mögen aber beispielsweise auch die Kombination Tomate und Käse.
    }

\end{recipe}
