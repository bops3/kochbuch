\begin{recipe}
[ % Optionale Eingaben
%    preparationtime = {\unit[30]{min}},
    bakingtime={\unit[30]{min}},
    bakingtemperature={\Fanoven\ \unit[200]{C}},
   portion = \portion{4},
    %calory,
    source = Sophie
]
{Quiche Lorraine}

\introduction{
    \# Rezept meiner Mutter, bisher unübertroffen von allem, was meine französischen Freundinnen gebacken haben.
}

\graph{
    big = sophie/Quiche,
}

\ingredients[11]
{% Zutatenliste
    \unit[200]{g} & Mehl \\
    \unit[100]{g} & Butter \\
    \unit[\nicefrac{1}{2}]{TL} & Salz \\
     & Beliebige Dinge für die Füllung (s. Tipp) \\
    4 & Eier \\
    \unit[\nicefrac{1}{4}]{L} & Sahne \\
    \unit[125]{g} & Geriebener Käse \\
    \unit[\nicefrac{1}{2}]{TL} & Weißer Pfeffer \\
    \unit[\nicefrac{1}{2}]{TL} & Salz
}

\preparation
{ % Schrittweise Zubereitung
    \\
Für den Teig Mehl, Salz und Butter in Flocken mit 5 EL Wasser rasch zu einem geschmeidigen Teig verarbeiten. In Pergamentpapier einschlagen und in den Kühlschrank stellen

Den Backofen auf 200 °C vorheizen. Eine Springform mit Butter ausstreichen und mit Mehl bestäuben. Den Teig ausrollen und so in die Springform legen, dass der Teig einen Rand bildet. 

Den Teig für ca. 10 Minuten vorbacken (es hilft, den Teig mit trockenen Bohnen auf Pergamentpapier zu beschweren, damit die Seiten nicht einfallen).

Für die Füllung die Eier trennen. Die Eigelbe mit Sahne, Salz, Pfeffer und eventuellen anderen Gewürzen (z.B. Muskat) verquirlen. Den Käse und die sonstige Füllung (Gemüse o.ä.) unterziehen. Die Eiweiße mit etwas Salz steif schlagen und unter die Eigelbmasse heben. 

Die Füllung auf den vorgebackenen Teig geben und die Quiche für mindestens 30 Minuten im Backofen bei 200 °C backen.
}

\hint
    {% Hinweise
    Für die Füllung: Klassischerweise Speck/Schinken (200 g), ansonsten z.B. gedünstetes Gemüse (mediterran mit Paprika, Zucchini und getrockneten Tomaten ; oder Lauch mit Karotten). Ich mag die Kombination Spinat + Schafskäse sehr gerne.
    
    Meine französischen Freund:innen haben das Trennen der Eier meist weggelassen, um Zeit zu sparen. Ich finde, die Quiche wird durch den Eischnee aber sehr viel fluffiger und insgesamt leckerer.

    Ich sollte häufiger fotografieren, was ich koche/backe. Weil ich das nicht getan habe hier ein Internet-Foto zur Inspiration.
    }

\end{recipe}
