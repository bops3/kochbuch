\begin{recipe}
[ % Optionale Eingaben
%    preparationtime = {\unit[30]{min}},
%    bakingtime={\unit[10]{min}},
%   bakingtemperature={\Fanoven\ \unit[250]{C}},
%    portion = \portion{4},
    %calory,
    source = Sophie
]
{Französische Zwiebelsuppe}

\graph{
    big = sophie/Zwiebelsuppe.png,
}

\ingredients[8]
{% Zutatenliste
    \unit[300]{g} & Zwiebeln \\
    2 & Knoblauchzehen \\
    \unit[1]{EL} & Butterschmalz \\
    \unit[1]{EL} & Mehl \\
    \unit[1 \nicefrac{1}{4}]{L} & Brühe \\
     & Salz und Pfeffer  \\
     4 & Brotscheiben  \\
     \unit[100]{g} & Geriebener Käse
}

\preparation
{ % Schrittweise Zubereitung
    \\
Die Zwiebeln in Ringe schneiden, Knoblauch fein hacken. Butterschmalz in einem Kochtopf zerlassen, Zwiebeln und Knoblauch dazugeben und goldgelb rösten.

1 EL Mehl hinzugeben und unter Rühren für ca. 5 Minuten durchschwitzen. Langsam mit der Brühe aufgießen und ca. 20-25 Minuten aufkochen lassen. Kräftig mit Salz und Pfeffer würzen.

Das Brot würfeln und rösten. Die Suppe in feuerfeste Tassen füllen, die Brotwürfel dazugeben und mit Käse überstreuen. Im vorgeheizten Backofen bei 250 °C für 10 Minuten backen.

}

\hint
    {% Hinweise
    Geht auch ohne das Überbacken, aber Käse macht einfach viele Dinge besser.

    Das Bild ist aus dem Internet geklaut, aber selbst gemacht sieht genauso aus!
    }

\end{recipe}
