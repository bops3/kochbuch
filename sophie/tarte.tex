\begin{recipe}
[ % Optionale Eingaben
%    preparationtime = {\unit[30]{min}},
    bakingtime={\unit[50]{min}},
    bakingtemperature={\Fanoven\ \unit[180]{C}},
    %calory,
    source = Sophie
]
{Zitronen-Meringue-Tarte}

\introduction{
    \# Rezept meiner französischen Mitbewohnerin. Sie hat gerne die geniale Aufteilung der getrennten Eier in den verschiedenen Tarte-Komponenten betont.
}

\graph{
    big = sophie/Zitronentarte,
}

    \ingredients[19]
{% Zutatenliste
    \textit{Mürbteig} & \\%evtl anzupassen
    \unit[250]{g} & Mehl \\
    \unit[125]{g} & Butter \\
    \unit[70]{g} & Zucker \\
    2 & Eigelbe \\
    ca. \unit[50]{mL} & Wasser \\
    \unit[1]{Prise} & Salz \\
    \textit{Zitronencreme} & \\%evtl anzupassen
    2 & Bio-Zitronen \\
    \unit[180]{g} & Zucker \\
    2 & Eier \\
    2 & Eigelbe \\
    \unit[200]{mL} & Saure Sahne \\
    \unit[1]{Prise} & Salz \\
    \textit{Meringue-Haube} & \\%evtl anzupassen
    4 & Eiweiße \\
    \unit[150]{g} & Puderzucker \\
    \unit[1]{Prise} & Salz
}

\preparation
{ % Schrittweise Zubereitung
    \\
Den Teig vorbereiten: Den Ofen auf 180 °C (Umluft) vorheizen. Die Eigelbe und Zucker mit etwas Wasser verquirlen. Mehl und Butter mit den Fingern mit den Fingern leicht zusammenkneten, bis eine sandartige Konsistenz erreicht wird. Die flüssige Mischung zu Mehl und Butter geben und den Teig zu einem Ball kneten. Den Teig dann ausrollen, in die Springform legen und den Teigboden einige Mal mit einer Gabel anstechen. Für 10 Minuten backen (den Teig mit Bohnen beschweren ; der Teig sollte durch das Backen nicht braun werden oder nur leicht).

Die Creme zubereiten: Zwei Eier mit zwei weiteren Eiweißen, der sauren Sahne und dem Zucker verrühren. Den Saft von zwei Zitronen und deren abgeriebene Schale dazugeben. Den Teig aus dem Ofen holen und die Bohnen entfernen, dann die Creme auf den Teig geben und für 30 Minuten backen.

In der Zwischenzeit die Meringue vorbereiten: Die vier verbleibenden Eiweiße mit einer Prise Salz steif schlagen. Dann den Puderzucker nach und nach unter Rühren zugeben, bis die gewünschte Konsistenz erreicht ist.

Wenn die Creme gestockt ist (evtl. etwas länger als 30 Minuten backen), die Meringue auf der Tarte verteilen und für weitere 10 Minuten backen, bis die Meringue leicht bräunlich wird. Für eine härtere Meringue sollte die Tarte weitere 30 Minuten im Ofen bleiben.

}

\hint
    {% Hinweise
    Klingt nach Arbeit, lohnt sich aber, versprochen!
    
    Der Zucker in allen drei Komponenten kann nach Belieben reduziert werden.
    }

\end{recipe}
