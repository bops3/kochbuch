
\begin{recipe}
[ % Optionale Eingaben
    preparationtime = {\unit[20]{min}},
    %bakingtime={},
    %coolingtime={\unit[30]{min}},
    %bakingtemperature={},
    portion = \portion{4},
    %calory,
    source = Paula
]
{Debesmanna -- Himmelsgrieß}

\introduction{
    \# lustige Konsistenz
}

\graph{
    small = paula/dm1,
    big = paula/dm3,
}

\ingredients
{% Zutatenliste
    \unit[500]{ml} & Cranberry Nektar \\
    \unit[5]{gestrichene EL} & Grieß \\
     & (pflanzliche) Milch
}

\preparation
{ % Schrittweise Zubereitung
    \\
    Den Cranberry Nektar aufkochen. Den Grieß langsam dazugeben und gut einrühren, damit sich keine Klümpchen bilden. 
    Unter Rühren bei geringer Hitze für etwa 5 Minuten köcheln lassen.

    Den fertig gekochten Grießbrei abkühlen lassen. Dann mit einem Schneebesen oder Handrührgerät aufschäumen. 
    Er soll deutlich heller werden und sein Volumen vergrößern.

    Zum Servieren kalte (pflanzliche) Milch in Schüsseln geben und mit einem Löffel Grießwölkchen auf der Milch schwimmen lassen.
}

\hint
    {% Hinweise
    Andere Säfte, Marmeladenreste mit Wasser gemischt oder frische Früchte (diese müssen dann erst gekocht und püriert werden) können auch genutzt werden.

    Die Bilder wurden nicht alle von einer glücklichen KI generiert.
    }

\end{recipe}
