
\begin{recipe}
[ % Optionale Eingaben
    preparationtime = {\unit[45]{min}},
    %bakingtime={\unit[40]{min}},
    %bakingtemperature={\Fanoven\ \unit[180]{C}},
    portion = \portion{6},
    %calory,
    source = Paula
]
{Kürbissuppe mit Kokosmilch und Ingwer}

\introduction{
    \# lecker trotz Ingwer
}

\graph{
    small = paula/k4,
    big = paula/k1,
}

\ingredients
{% Zutatenliste
    \unit[800]{g} & Hokkaido (geputzt) \\
    \unit[600]{g} & Möhren (geschält) \\
    \unit[5]{cm} & Ingwer \\
    1 & Zwiebel \\
    \unit[2]{EL} & (vegane) Butter \\
    \unit[1]{l} & Gemüsebrühe \\
    \unit[500]{ml} & Kokosmilch \\
     & Salz, Pfeffer \\
     & Sojasauce \\
     & \\
    \unit[1]{Zitrone} & Saft \\
    zum Servieren & Crème Vega oder Fraîche \\
    optional & Koriandergrün \\  
}

\preparation
{ % Schrittweise Zubereitung
    \\
    Kürbis, Möhren, Ingwer und Zwiebel schälen und würfeln, in der Butter andünsten.

    Mit der Brühe aufgießen und in etwa 15 - 20 Minuten weich kochen. Dann sehr fein pürieren.

    Die Kokosmilch unterrühren, mit Salz, Pfeffer, Sojasauce und Zitronensaft abschmecken und noch mal erwärmen. Mit Korianderblättchen garniert servieren.
}

\hint
    {% Hinweise
    Lass den Koriander weg! Der schmeckt zwar nicht nach Seife, aber trotzdem eklig.

    Die Bilder wurde von einer glücklichen KI generiert.
    }

\end{recipe}
