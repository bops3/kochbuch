
\begin{recipe}
[ % Optionale Eingaben
    preparationtime = {\unit[45]{min}},
    bakingtime={\unit[40]{min}},
    bakingtemperature={\Fanoven\ \unit[180]{C}},
    portion = \portion{4},
    %calory,
    source = Paula
]
{Vegane Lasagne mit Béchamelsauce}

\introduction{
    \# Gäste können mitschnippeln (bei Gemüseerweiterung)
}

\graph{
    small = paula/l2,
    big = paula/l1,
}

\ingredients[26]
{% Zutatenliste
    \textit{Hacksauce} & \\
    \unit[150]{g} & Sojagranulat \\
    \unit[1]{l} & Gemüsebrühe \\
    1 & Zwiebel \\
    2 & Möhren \\
    \unit[1]{EL} & Sojasauce \\
    \unit[6]{EL} & Tomatenmark \\
    \unit[1]{EL} & Agavendicksaft \\
    \unit[100]{ml} & Rotwein \\
    \unit[500]{g} & gestückelte Tomaten aus der Dose \\
    \unit[0.5]{TL} & getr. Basilikum \\
    \unit[0.5]{TL} & getr. Oregano \\
    \unit[0.25]{TL} & Cayennepfeffer \\
    \unit[0.5]{TL} & Paprikapulver \\
     & Pflanzenöl zum Braten \\
     & Salz, Pfeffer \\
    \textit{Béchamelsauce} & \\
    \unit[3]{EL} & Weizenmehl \\
    \unit[3]{EL} & (vegane) Butter \\
    \unit[250]{ml} & (pflanzliche) Milch \\
    \unit[1]{EL} & Hefeflocken \\
     & Salz, Pfeffer, Muskatnuss\\
    \textit{Zusätzlich} & \\
     & Lasagneplatten \\
    optional & veganen Reibekäse \\
}

\preparation
{ % Schrittweise Zubereitung
    \\
    Für die Tomaten-Hack-Schicht getrocknetes Sojagranulat in einer hitzefesten Schüssel oder einem Topf mit heißer Gemüsebrühe übergießen und ca. 10-15 Minuten einweichen lassen. Anschließend abgießen, so viel Flüssigkeit wie möglich aus dem Granulat herausdrücken und großzügig mit Salz und Pfeffer würzen. Alternativ könnt ihr diesen Schritt auch überspringen, wenn ihr einen Hackfleisch-Ersatz habt. den man nur noch anbraten muss.

    Während das Sojagranulat einweicht, Zwiebel schälen und fein würfeln. Möhre raspeln oder ebenfalls in kleine Würfel schneiden.

    Pflanzenöl in einer Pfanne über mittlerer bis hoher Hitze erwärmen und das Sojagranulat darin ca. 5 Minuten scharf anbraten, damit es gebräunt wird. Sojasauce dazugeben und weitere 5 Minuten anbraten.

    Zwiebel und Möhre in die Pfanne geben und alles gemeinsam über mittlerer Hitze ca. 4-5 Minuten anbraten. Danach Tomatenmark und Agavendicksaft dazugeben und ca. 3 Minuten köcheln lassen.

    Mit Rotwein ablöschen und ca. 5 Minuten einköcheln lassen, bevor ihr die gestückelten Tomaten dazugebt. Gewürze dazugeben, mit Salz und Pfeffer abschmecken und die Sauce auf kleiner Hitze köcheln lassen, bis die Béchamelsauce fertig ist.

    Für die Béchamelsauce vegane Butter in einem kleinen Topf schmelzen lassen und Mehl einrühren. Ca. 1 Minute über kleiner Hitze anschwitzen, danach pflanzliche Milch langsam dazugeben und immer schön rühren, damit sich keine Klumpen bilden. Mit Hefeflocken, Salz und Pfeffer abschmecken. Falls die Béchamelsauce zu dick wird, könnt ihr noch etwas mehr pflanzliche Milch dazugeben.

    Backofen auf 180°C vorheizen. Es wird Zeit für eure Auflaufform. Beginnt mit einer dünnen Schicht Tomate-Sojahack, legt dann trockene Lasagneplatten darüber, anschließend wieder eine Schicht Tomaten-Sojahack-Sauce und bedeckt das ganze mit einer Schicht Béchamelsauce. Jetzt wieder Lasagneplatten, Tomaten-Sojahack-Sauce, Béchamelsauce und immer so weiter, bis ihr alles aufgebraucht habt und oben bei einer letzte Schicht Béchamelsauce ankommt. Optional könnt ihr jetzt noch veganen Käse (oder auch Hefeschmelz, oder nur noch ein paar Hefeflocken) darüberstreuen, aber auch ohne schmeckt die Lasagne herrlich. Wer veganen Streukäse nimmt, sollte ihn vorher mit ein bisschen Öl in einer kleinen Schüssel verrühren. Dadurch wird er fettiger und schmilzt besser im Ofen.

    Die Lasagne bei 180°C ca. 40 Minuten backen, oder eben bis die Lasagneplatten weich sind. Danach aus dem Backofen nehmen und die Lasagne vor dem Anschneiden noch ca. 5 Minuten stehen und anziehen lassen.
}

\hint
    {% Hinweise
    Meistens nehme ich (grob) passierte Tomaten und mehr als im Rezept steht (2 passierte Tomaten Flaschen in der Regel). Mehr verschiedenes Gemüse ist auch gut; Mais passt sehr gut in die Lasagne, aber auch Zucchiniwürfel oder Pilze (auch, wenn die kein Gemüse sind).

    Die Bilder wurde von einer glücklichen KI generiert.
    }

\end{recipe}
